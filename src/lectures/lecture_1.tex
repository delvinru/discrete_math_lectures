\section{Дискретные функции и их представление. Индуктивное определение формулы. Полные системы. Критерий полноты.}

\opri Дискретной функцией называется любая функция, отображающая конечное множество $A$ в конечное множество $B$.

Область определения дискретной функции часто представляется в виде декартового произведения множеств относительно небольшой мощности.

Если \map{f}{A}{B} - дискретная функция и $A = A_1 \times \dots \times A_n$, то $f$ обозначают следующим образом $f(x_1;\dots;x_n)$ и называют
дискретной функцией от $n$ переменных $x_1,\dots,x_n$. При этом $x_i$ принимает всевозможные значения из $A_i$.
Если $A_1 = \dots = A_n = B$ и $B = \{0, 1\}$, то $f$ называется булевой функцией.

\opr Обозначим далее $\Omega = \{0, 1\}$, тогда булевой функцией от $n$ переменных называется любое отображение \map{f}{\Omega^n}{\Omega}.

0-местными булевыми функциями будем называть элементы $0, 1 \in \Omega$.

\note Существуют функции $k$ - значной логики.

Обозначать булеву функцию будем $f(x_1;\dots;x_n)$ или $f(\vec{x})$, если количество переменных известно из контекста.

\opr Если $f(x_1;\dots;x_n)$ - булева функция и $\vec{\alpha} = (a_1; \dots; a_n) \in \Omega^n$, то образ $\vec{\alpha}$ при отображении
$f$ называют значением функции $f$ на наборе $\vec{\alpha}$. Обозначение: $f(\vec{\alpha})$.

\opr Если рассматривать 0 и 1 как числа $\in \N_0$, то для набора $\vec{\alpha} = (a_1; \dots; a_n)$ обозначим
$||\vec{\alpha}|| = a_1 + \dots + a_n$ - вес вектора $\vec{\alpha}$.

$\widetilde{a} = \sum\limits_{i=1}^{n} a_i 2^{n-i}$ - лексикографический порядок.

\example
$$
\vec{\alpha} = (1; 1; 0; 1) \Rightarrow ||\vec{\alpha}|| = 1 + 1 + 0 + 1 = 3.
$$

Естественным образом задания является табличный, при этом координата $i$-вектора $f^{\downarrow}$ соответствует значению
$f(\vec{\alpha})$, где $\widetilde{a} = i$.

\example
$$
\begin{matrix}
    x_0 & x_1 & f^{\downarrow} \\
    0 & 0 & 0 \\
    0 & 1 & 1 \\
    1 & 0 & 1 \\
    1 & 1 & 1 \\
\end{matrix}
$$