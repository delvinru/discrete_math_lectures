\section{Дискретные функции и их представление. Индуктивное определение формулы. Полные системы. Критерий полноты.}

\opri Дискретной функцией называется любая функция, отображающая конечное множество $A$ в конечное множество $B$.

Область определения дискретной функции часто представляется в виде декартового произведения множеств относительно небольшой мощности.

Если \map{f}{A}{B} - дискретная функция и $A = A_1 \times \dots \times A_n$, то $f$ обозначают следующим образом $f(x_1;\dots;x_n)$ и называют
дискретной функцией от $n$ переменных $x_1,\dots,x_n$. При этом $x_i$ принимает всевозможные значения из $A_i$.
Если $A_1 = \dots = A_n = B$ и $B = \{0, 1\}$, то $f$ называется булевой функцией.

\opr Обозначим далее $\Omega = \{0, 1\}$, тогда булевой функцией от $n$ переменных называется любое отображение \map{f}{\Omega^n}{\Omega}.

0-местными булевыми функциями будем называть элементы $0, 1 \in \Omega$.

\note Существуют функции $k$ - значной логики.

Обозначать булеву функцию будем $f(x_1;\dots;x_n)$ или $f(\vec{x})$, если количество переменных известно из контекста.

\opr Если $f(x_1;\dots;x_n)$ - булева функция и $\vec{\alpha} = (a_1; \dots; a_n) \in \Omega^n$, то образ $\vec{\alpha}$ при отображении
$f$ называют значением функции $f$ на наборе $\vec{\alpha}$. Обозначение: $f(\vec{\alpha})$.

\opr Если рассматривать 0 и 1 как числа $\in \N_0$, то для набора $\vec{\alpha} = (a_1; \dots; a_n)$ обозначим
$||\vec{\alpha}|| = a_1 + \dots + a_n$ - вес вектора $\vec{\alpha}$.

$\widetilde{a} = \sum\limits_{i=1}^{n} a_i 2^{n-i}$ - лексикографический порядок.

\example
$$
\vec{\alpha} = (1; 1; 0; 1) \Rightarrow ||\vec{\alpha}|| = 1 + 1 + 0 + 1 = 3.
$$

Естественным образом задания является табличный, при этом координата $i$-вектора $f^{\downarrow}$ соответствует значению
$f(\vec{\alpha})$, где $\widetilde{a} = i$.

\example
$$
\begin{matrix}
    x_0 & x_1 & f^{\downarrow} \\
    0 & 0 & 0 \\
    0 & 1 & 1 \\
    1 & 0 & 1 \\
    1 & 1 & 1 \\
\end{matrix}
$$

\utv $|F_2(n)|= 2^{2^n}$.

\opr Весом булевой функции $f$ называют величину $||f|| = |\{\vec{\alpha} \in \Omega^n \mid f(\vec{\alpha}) = 1\}|$.

\opr Функция от $n-1$ переменной, определяемая равенством $\varphi(a_{i_n}; \dots; a_{i_n}) = f'(a_1; \dots; a_{i-1}; b; a_{i+1}; \dots; a_n)$, называется
функцией полученной из $f'$ фиксацией $i$-ой переменной значением $b$.

Обозначением $\varphi = f_i^b(x_1; \dots; x_n)$, аналогично фиксация $k$ переменных значениями $b_1,\dots, b_k : \varphi = f_{i_1; \dots; i_n}^{b_1; \dots; b_k}(x_1; \dots; x_n)$.

Общее название таких функци $\varphi$ - подфункции $f$. 

Если $f(a_1; \dots; a_{i-1};0;a_{i+1}; \dots; a_n) = f(a_1; \dots; a_{i-1};1;a_{i+1}; \dots; a_n)$, то
переменная $x_i$ называется несущественной переменной функции $f$, в противном случае - существенной.

\opr Пусть $x_i$ -несущественная(\emph{фиктивная}) переменная функции $f$, $g$ получена из $f$ фиксацией $x_i$ любой константой, тогда говорят, что $g$
получена удалением из $f$ несущественной переменной $x_i$, а $f$ получена из $g$ добавлением фиктивной переменной $x_i$.

Пусть задано множество функций $\K = \{f_i : i \in I\}$ и множество символов переменных $X = \{x_1; \dots; x_n\}$.

\opr \begin{enumerate}
    \item Любой символ переменной есть формула над классом $\K$.
    \item Если $f_j$ - символ $m$ - местной функции из $\K$, а $A_1, \dots, A_m$ - формулы над $\K$, то $f_j(A_1; \dots; A_m)$ - формула над $\K$.
    \item Других формул нет.
\end{enumerate}

Множество формул над $\K$ обозначается $\Phi(\K)$. При $m=0$ формула есть символ над $\K$, т.е. константа.

\opr Число символов функций из $\K$, встречающихся в формуле $A$ назовем рангом формулы $A$. Обозначение: $r(A)$.

\opr \begin{enumerate}
    \item Подформула формулы $x_i$ - только она сама.
    \item Подформулы $f_j(A_1; \dots; A_n)$ - на сама и все подформулы формулы $A_1; \dots; A_n$.
\end{enumerate}

\opr Пусть $A$ - произвольная формула, в ее записи присутствует только переменные $x_{i_1},\dots, x_{i_k}$. Набор $x_{j_1},\dots, x_{j_m}$ называется
допустимым, если $\{x_{i_1},\dots, x_{i_k}\} \subseteq \{x_{j_1},\dots, x_{j_m}\}$.

Каждой формуле при фиксированном допустимом наборе $(x_1; \dots; x_n)$ сопоставляется по следующему правилу:
\begin{enumerate}
    \item Если $A$ есть $x_i$, то ей сопоставляется функция $f$, значения которой определяются равенством $f(a_1;\dots;a_n) = a_i, (a_1;\dots;a_n) \in \Omega^n$.
    \item Если $A$ есть $f_j(A_1;\dots; A_m)$ и формулам $A_1, \dots, A_m$ сопоставлены функции $\varphi_1(x_1;\dots;x_n); \dots;\varphi_m(x_1;\dots;x_n)$, то формуле $A$ сопоставляется
функция $f$, значения которой определяются равенством $f(a_1;\dots;a_n) = f_j(b_1;\dots;b_n)$, где $b_\zeta = \varphi_\zeta(a_1;\dots;a_n), \zeta \in \overline{1, n}$.
\end{enumerate}

\opr Формулы $A$ и $B$ равносильны, если они представляют одну и ту же функцию на любом допустимом наборе. Обозначение: $A\equiv B$.

\opr Пусть $A$ - произвольная формула над классом $\K = ( \text{\&}, \vee, \not )$. Двойственной к $A$ называется формула полученная из $A$ заменой $\text{\&} \leftrightarrow \vee$. Обозначение: $A^*$.

\thr $A^*(x_1; \dots; x_n) = \not{A(\not{x_1}; \dots; \not{x_n})}$.

\sled $A \equiv B \Leftrightarrow A^* \equiv B^*$.

\opr Замыканием системы $\K$ булевых функций называют множество всех булевых функций представимых формулами над $\K$. Обозначение: $[\K]$.

\utv \begin{enumerate}
    \item $\K \subseteq [\K]$
    \item $\K_1 \subseteq \K_2 \Rightarrow [\K_1] \subseteq [\K_2]$
    \item $[[\K]] = [\K]$
\end{enumerate}

\opr Система $\K$ называется полной, если (замыкание) $[\K] = F_2$.