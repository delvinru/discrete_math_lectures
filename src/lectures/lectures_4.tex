\section {Представление дискретных функций в базисах функциональных пространств. Алгоритм БПФ. }

\opr Пусть K - произвольное поле, 0 и 1 - нуль и единица поля K  соответствует псевдобулевой функции от n переменных называется произвольное отображение $f:\{0,1\}^{n} \longrightarrow K.$
Обобщение $ GF(p)^n \longrightarrow K$.
Множества таких функций будем называть $Kp(n)$. На $Kp(n)$ естественным образом задаются операции + и \cdot

\utv $Kp(n)$ - векторное пространство над K размерности $p^n$

\thr Множеству всех различных гомоморфизмов $\varphi (GF(p)^n,+) \longrightarrow (\mathbb{C},*)$ состоит из $p^n$ различных гомоморфизмов $X_\alpha$; $\alpha = (\alpha_1,...,\alpha_n) \in GF(p^n)$, каждый из которых однозначно определяется своим действием на вектора стандартного базиса $e_j$ следующим образом $X_\alpha(e_j) = exp(\frac{2\pi i}{p}\cdot \alpha_i).$

\utv Для любых $\alpha,\beta \in GF(p)^n$ верно $\frac{1}{p^n}\sum_{\gamma=1 \in GF}^n X_\alpha (\gamma) \overline {X_\beta (\gamma )} = \delta_{\alpha,\beta}$

\begin{equation*}
    \begin{cases}
        1,$\alpha=\beta$
        \\
        0,$\aplha \ne \beta$
    \end{cases}
\end{equation*}

\thr ${X_\alpha,\alpha \in (GF(p))^n}$ - базис $\mathbb{C}_p(n)$

\opr Разложение произвольной функции $f \in C_p(n)$ по базису характерен: ${X_\alpha ;\alpha \in (GF(p))^n}: f(\overline{x}) =  \sum_{\alpha \in GF(p)^n} C_\alpha^fX_\alpha(\overline{x})$ называется разложением $f$ в ряд Фурье, соответствующий набору $\alpha$. Комплексное число $C_\alpha^f$ - коэффициент Фурье.

\opr Преобразование из $C_p(n)$ в $\mathbb{C}^{p^{n}}$, ставящее в соответствие каждой функции ее коэффициент фурье, будем называть преобразование Фурье.

\utv
\\
1)Пусть $\gamma \in GF(p)^n.$ Тогда $C_\gamma^f = \frac{1}{p^n}\sum_{\beta \in GF} f(\beta) \overline{X_\alpha(\beta)}$
\\
2) Пусть $f$ - б/ф, тогда $C_\gamma^f = \frac{1}{2^n}\textbar(f(\overline{x})\textbar$

В некоторых случаях вместо функции $f$ удобнее свойства функции $F(\overlinex) = (-1)^{f(\overlinex)}.$ Коэф. Фурье такой функции называется коэффициентом Уолта-Адилора второго рода функции $f(x)$. Обозначается $C_\alpha^f = W_\alpha^f$.

\utv Свойства:
\\
1) $W_\alpha^f = 1 - \frac{1}{2^{n-1}} \textbar\textbar f(\overlinex) \oplus <\alpha,\overline{x}>\textbar\textbar\textbar\textbar$
\\
2)
\begin{equation*}
    W_\alpha^f =
    \begin{cases}
        $-2C_\alpha^f$ &\text{ $\alpha \pm \overline{0}$}\\
        $1-2C_\alpha^f$ &\text{ $\alpha = \overline{0}$}
    \end{cases}
\end{equation*}\\
3)$\sum_{\alpha \in \Omega} W_\alpha^f = (-1)^{f(\overline{0})} $
\\
4)$\sum_{\alpha \in \Omega_2^n} (W_\alpha^f)^2 = 1$
\\
5)$\frac{1}{2^{\frac{n}{2}}} \leq max_{\alpha \in \Omega^n} \textbar W_\alpha^f \textbar$

Зафиксируем некоторую обратимую $2^n*2^n$ матрицу $A$ над полем $K$. Пусть $f^+$ - вектор столбцов значений $f$ из $K_2(n)$. $\overline{f^+} = A^-1f^+.$ Тогда задано биективное отображение из $K_2(n)$ в $K^{2^{n}}.$ Вектор $\overline{f^+} $ - представление функции $f$, если столбцы матрицы $A$ занумеровать наборами из $\Omega_2^n$, то $f^+ = \sum_{\alpha \in \Omega_2^n} g_\alpha^+\overline{f}(\alpha)$. Каждый столбец $g_\alpha^+$ есть задание которой функции из $K_2(n)$,$A$ - невырожденная. ${g_\alpha} \alpha \in \Omega_2^n$ - базис $K_2(n)$.

\opr Пусть $A$ и $B$ - матрицы над размеров $m \times n$ и $n \times n$ над полем $K$ соответственно. Тензорным произведением матриц $A$ и $B$ называется матрица $С = A \otimes B$.
\begin{equation*}
    C =
    \begin{pmatrix}
        \alpha_{11}\beta& \alpha_{12}\beta&  ...&  \alpha_{1m}\beta\\
        ...& ...& ...& ...\\
        ...& ...& ...& \alpha_{mm}\beta
    \end{pmatrix}
\end{equation*}\\

Размерность $mn\times mn$.

\utv При условии $m = n$\\
1) $A \otimes (B \otimes C) = (A \otimes B) \otimes C$\\
2) $(A + B) \otimes C = A \otimes C + B \otimes C $\\
3) $A \otimes (B + C) = A \otimes B + A \otimes C $\\
4) $A,C  \in K_{m,m}, B,D \in K_{n,n} (A \otimes B)(C \otimes D) = AC \otimes BD$\\
5) Тензорное произведение обратимо $\Leftrightarrow$  $A$ и $B$ обратимы, причем $(A \otimes B)^{-1} = A^{-1} \otimes B^{-1}$

\utv Лемма 1\\
Пусть матрица размера $2^n \times 2^n$ над $K$ и $A = B \otimes A`$, где B = \begin{pmatrix}
                                                                                 a & b\\
                                                                                 c & d\\
\end{pmatrix}

$a,b,c,d \in K$, а $A$ - матрица размером $2^{n-1} \times 2^{n-1}$ причем обе матрицы $B$ и  $A`$ невырожденные. Пусть столбцы матриц $A$ и $A`$ задают базисы функциональных пространств $K_2(n),K_2(n-1)$, функции из которых обозначаются $g_\alpha$ и $g_\alpha`$ соответственно. Тогда для любых $\alpha \in \Omega_2^{n-1}$ верно:\\
$g(0,\alpha`) = (a\overline{x_1} + cx_1)g_\alpha(x_2,...,x_n)$\\
$g(1,\alpha`) = (b\overline{x_1} + dx_1)g_\alpha`(x_2,...,x_n)$

\thr Пусть A - тензорное произведение матриц $B_j \in K_{2\times2}^*$ вида
\begin{pmatrix}
    $a_i$ & $b_i$\\
    $c_i$ & $d_i$\\
\end{pmatrix},
то есть $A = \otimes \Pi_{i = 1}^n \beta_i$,тогда базисная функция $g_\omega$, соотвествующая $A$ и занумерованная набором $\omega = (\omega_0,...,\omega_n)$ имеет вид $g_\omega = \Pi_{i = 1}^n \overline{\omega_i}(a_i\overline{x_1} + c_ix_1) + \omega_i(b_i\overline{x_1} + d_ix_1)$\\

\thr Пусть $B$ - невырожденная матрица размера $2 \times 2$  над $K$; $A = B^{[n]}$ - тензорная степень. Тогда существует алгоритм вычисления $\overlive{f^+}$ по вектору $f^+$, имеющий сложность $\overline{0}(n2^n$ операций поля $K$)\\
Док-во:\\
Пусть $B^{-1} = $ \begin{pmatrix}
                      $a$ & $b$\\
                      $c$ & $d$\\
\end{pmatrix}. Из свойств тензорного произведения матриц вытекает, что $A^{-1} = (B^{-1})^{[n]}= D_n \cdot D_{n-1} \cdot ... \cdot D_1$, где $D_i$ - матрица вида:\\
$D' = (E_2^{[n-i]}\otimes $
\begin{pmatrix}
    a & b\\
    c & d\\
\end{pmatrix}$\otimes E_2^{[i-1]})$. Обозначим $f_0^+ = f^+$ и $\forall i \in \overline{1,n} f_i^+ = D_if_{i-1}^+$.
Тогда $\overline{f^+} = f_n^+$
Покажем, что каждое из умножений $D_i$ на $f_{i-1}^+$ может быть выполнено за $0(2^n)$ операций поля $K$. Тогда общее количество операций, необходимое для вычисления $\overline{f^+}$ по $f^+$ будем составлять $0(n2^n)$ операций.\\

\begin{equation*}
    $D = (E_{2^{n-i}} \otimes$
    \begin{pmatrix}
        $aE_{2^{i-i}}$ & $bE_{2^{i-i}}$\\
        $cE_{2^{i-i}}$ & $dE_{2^{i-i}}$\\
    \end{pmatrix}) = \\

    \begin{pmatrix}
        D_i & ... & 0\\
        ... & ... & ...\\
        0 & ...  & D_i\\
    \end{pmatrix},
\end{equation*}\\
где $D_i$ - матрица размера $2^i \times 2^i$ вида
\begin{pmatrix}
    $aE_{2^{i-i}}$ & $bE_{2^{i-i}}$\\
    $cE_{2^{i-i}}$ & $dE_{2^{i-i}}$\\
\end{pmatrix}.

Пусть теперь $X^+$ произвольный  вектор длины $2^n$ над полем $K$. Опишем алгоритм умножения $D_i$ на $X^+$.\\
1) Разобьем $X^+$ на $2^{n-1}$ частей длины $2^i$, тогда
\begin{equation*}
    X^+ =
    \begin{pmatrix}
        x_1^+\\
        ...\\
        x_{2^{n-i}}^+\\
    \end{pmatrix}
\end{equation*}\\,таким образом получаем:
\begin{equation*}
    D_iX^+ =
    \begin{pmatrix}
        D_ix_1^+\\
        ...\\
        D_ix_{2^{n-i}}^+\\
    \end{pmatrix}
\end{equation*}
2) Каждый из векторов $X_j^+$ разбиваем на 2 подвектора равной длины.

\begin{equation*}
    X_j^+ =
    \begin{pmatrix}
        x_{j_0}^+\\
        x_{j_1}^+\\
    \end{pmatrix}
\end{equation*}. Тогда

\begin{equation*}
    D_iX_j^+ =
    \begin{pmatrix}
        $aE_{2^{i-i}}$ & $bE_{2^{i-i}}$\\
        $cE_{2^{i-i}}$ & $dE_{2^{i-i}}$\\
    \end{pmatrix}

    \begin{pmatrix}
        x_{j_0}^+\\
        x_{j_1}^+\\
    \end{pmatrix} =

    \begin{pmatrix}
        $aE_{2^{i-i}}x_{j_0}^+$ + $bE_{2^{i-i}x_{j_0}^+}$\\
        $cE_{2^{i-i}}x_{j_1}^+$ + $dE_{2^{i-i}}x_{j_1}^+$\\
    \end{pmatrix} =

    \begin{pmatrix}
        $ax_{j_0}^+$ + $bx_{j_0}^+}$\\
$cx_{j_1}^+$ + $dx_{j_1}^+$\\
\end{pmatrix}
\end{equation*}.



таким образом для вычисления $\overline{f^+}$ необходимо $0(n2^n)$ операций.