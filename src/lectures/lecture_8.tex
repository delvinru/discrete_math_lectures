\section{Симметричные блок-схемы}

\opr Семейство $(x_1, \dots, x_v)$ подмножеств множества $\mathfrak{X} = \lbrace x_1, \dots, x_v\rbrace$
называется симметричной сбалансированной неполной блок-схемой или $(v, k , \lambda)$ - конфигурацией, если:
\begin{align*}
	1)\;&0 < \lambda < k < v - 1 \text{ и } \forall \text{ } i, j  = \overline{1;v} \text{ }
	(\mathfrak{X}_1, \dots, \mathfrak{X}_v - \text{ блоки})\\
	2)\;&|\mathfrak{X}_i \cap \mathfrak{X}_j| = 
	\begin{cases}
		k, i = j \\
		\lambda, i \neq j
	\end{cases}
\end{align*}
\examplei
\begin{align*}
	x_1 &= \lbrace 2, 3 \rbrace \\
	x_2 &= \lbrace 1, 3 \rbrace \rightarrow k = 2, \lambda = 1, v = 3\\
	x_3 &= \lbrace 1, 2 \rbrace
\end{align*}

\prim $k = 2$, потому что при пересечении $\mathfrak{X}$-ов с одинаковыми индексами получим то же множество с тем же размером.
$\lambda = 1$, потому что при пересечении $\mathfrak{X}$-ов с разными индексами получим множество размером 1.
$v = 3$ - количество элементов.

Матрица инцидентности $A = (a_{ij})_{v \times v} \mid a_{ij} = $
$\begin{cases}
	1, \mathfrak{X}_j \in \mathfrak{X}_i \\
	0, \mathfrak{X}_j \notin \mathfrak{X}_i
\end{cases}$

Её элементы обладают следующими свойствами:
\begin{enumerate}
	\item $\sum_{j=1}^v a_{ij} = \vert \mathfrak{X}_i \vert = k$
	\item $\sum_{\mu=1}^v a_{i\mu}a_{j\mu} = \vert \mathfrak{X}_i \cap \mathfrak{X}_j \vert = \lambda \text{, где} i\neq j$
	\item Скалярное произведение строки или столбца матрицы $A$
	\begin{equation*}
		(A_j; A_j) = 
		\begin{cases}
			k, i = j \\
			\lambda, i \neq j
		\end{cases}
	\end{equation*}
	\item $AJ = kJ$
	\item $AA^T = \lambda \overline{j} + (k - \lambda)E$
	\item $\det(AA^T) = (k + \lambda(v - 1))(k - \lambda)^{v - 1}$
	\item $k(k - 1) = \lambda(v - 1)$
	\item $JA = kJ$
	\item $AA^T = A^TA$
\end{enumerate}

\prim $J$ - матрица из всех единиц

\opr Дополнением $(v, k, \lambda)$ - конфигурации $( \mathfrak{X}_1, \dots, \mathfrak{X}_v)$
называется семейство $(Y_1, \dots, Y_v)$ таких подмножеств множества $\mathfrak{X} = \lbrace x_1, \dots, x_v\rbrace$,
что $Y_i = \mathfrak{X}\backslash\mathfrak{X}_i$

\utv
$(Y_1, \dots, Y_v)$ - $(v; v - k; v - 2k + \lambda)$ - конфигурация.

Рассмотрим код:
$K = \lbrace A_1, \dots, A_v,\tilde{A_1}, \dots, \tilde{A_v} \rbrace$, где $A_i$ строки матрицы инцидентности.

\utv
\begin{align*}
	& K - (v; d - 1; t) \text{ код, где } d - 1 \text{ обнаруживает ошибок, } t - \text{ не проявляет ошибок.}\\
	& d = min\lbrace \rho(A_i, A_j), \rho(\tilde{A_i}, \tilde{A_j})\rbrace = min\lbrace (2(k - \lambda));v - 2(k - \lambda) \rbrace\\
	& t \leq \frac{d - 1}{2}.
\end{align*}

\opr
Квадратная матрица $H_{n\times n}$ называется матрицей Адамара, если $h_{ij} \in \lbrace -1; 1 \rbrace$
и $HH^T = nE$.

\prop
\begin{enumerate}
	\item $\det(HH^T) = (detH)^2 = (\det(nE)) = n^n$.
	\item Любая матрица Адамара обратима над полем $\Q$.
	\item $\forall H$ - нормальная т.к. $H^TH = HH^T$.
	\item $(H_i, H_j) = $
	$\begin{cases}
		n, i = j \\
		0, i \neq j
	\end{cases}$
	- столбцы и строки попарно ортогональны
	\item Перестановка строк и столбцов переводит матрицу Адамара в матрицу Адамара.
	\item Умножение на $(-1)$ сохранит свойство \emph{быть матрицей Адамара}.
\end{enumerate}

\thr
Для существования матрицы Адамара необходимо выполнение условий:
\begin{enumerate}
	\item либо $n = 1,2$.
	\item либо $n \equiv 0\Mod{4}$.
\end{enumerate}

\thr (О связи матрицы Адамара с $(v, k, \lambda)$ - кодом)

Существует взаимнооднозначное соответствие между матрицей Адамара размеров $4\mu \times 4\mu$
и $(4\mu - 1; 2\mu - 1; \mu - 1)$

\proof
Рассмотрим некоторую нормализованную матрицу Адамара $H$ в ее последних $(n - 1)$ строках
и $(n - 1)$ столбцах. Выделим подматрицу $\tilde{H}$. Матрице $\tilde{H}$ сопоставим $\tilde{A} - 0;1$
матрицу $(n - 1)\times (n - 1)$. Элементы  матрицы $\tilde{A}$ определяются соотношением
$h_{ij} = (-1)^{a_{ij}}$.

Для каждой пары различных строк матрицы $\tilde{H}$ соответствующие столбцы матрицы $2\times (n - 1)$
имеют вид: 
$\underset{
\mu - 1
}{
\begin{pmatrix}
	1 \\
	1
\end{pmatrix}	
}$
$\underset{
\mu
}{
\begin{pmatrix}
	1 \\
	-1
\end{pmatrix}	
}$
$\underset{
\mu
}{
\begin{pmatrix}
	-1 \\
	1
\end{pmatrix}	
}$
$\underset{
\mu
}{
\begin{pmatrix}
	-1 \\
	-1
\end{pmatrix}	
}$
, где $\mu$ убранный $i$-й столбец.

Матрица $\tilde{A}$ - матрица инцидентности, если $v = 4\mu - 1, k = 2\mu, \lambda = \mu$.

Т.к. $n = 4\mu$ и $(\tilde{A}_i, \tilde{A}_j) = $
$\begin{cases}
	2\mu, i = j \\
	\mu, i\neq j
\end{cases}$

Рассуждения справедливы и в обратном порядке. А для каждой $(v, k, \lambda)$ конфигурации
можно построить соответствующий код. $\blacksquare$
\newpage
Пусть $p(f(x) = <a, x>) = \frac{1}{2} + \frac{\triangle_a^f}{2^n}$

\opr
Упорядоченный набор чисел $\triangle_a^f$ называется статистической структурой функции $f$.

\utv
\begin{enumerate}
	\item $\triangle_a^f = 2^{n - 1} - \vert\vert f(x) + <a, x> \vert\vert$
	\item $\triangle_a^f = 2^{n - 1} W_a^f$
	\item $\triangle_a^f = -\triangle_a^f$
	\item $\triangle_a^{f_1 \oplus f_2} = \frac{1}{2^{n - 1}} \sum_{\beta}\triangle_{\beta}^{f_1}\triangle_{a \oplus \beta}^{f_2}$
	\item Пусть $A$ - невырожденная над $GF(2),\text{ } \beta \in \Omega_2^n,\text{ } g(x) = f(Ax + \beta) \text{ }$\\
	$\triangle_{aA^T}^g = (-1)^{<a,b>}\triangle_a^f, a \in \Omega_2^n$

	\item Пусть $g(x) = f(x) \oplus <\gamma, x> \oplus d_0, $ где $\gamma \in \Omega_2^n, d_0 \in \Omega_2$.\\
	Тогда $\triangle_{a\oplus\gamma}^g = (-1)^{d_0}\triangle_a^f$
\end{enumerate}

Идея нахождения коэффициента $\triangle_a^f$ для булевой функции $(A_a^f)$:
\begin{enumerate}
	\item $\overrightarrow{f} \rightarrow (-1)^{\overrightarrow{f}}$.
	\item Преобразование $+-$ над половинками.
\end{enumerate}

\examplei
Вычислить коэффицент статической структуры для функции:
$$
	f = (1 0 0 1 0 1 1 0 1 0 0 1 0 1 1 0).
$$
Постром следующий вектор:
$$
	\vec{f}(x_1, x_2, x_3, x_4) = (1 0 0 1 \  0110\  1001\  0110).
$$

Далее строим вектор:
$$
(-1)^{\vec{f}} = ((-\!1 1 1\!-\!1) \  (1\!-\!1\!-\!11)\  (-\!111\!-\!1) \  (1\!-\!1\!-\!11)).
$$

Поделим его пополам. В левой части - сумма двух половинок. В правой части - разность первой и второй. (\emph{вычитаем из первой вторую})
\begin{align*}
	(-\!111\!-\!1) \ (1\!-\!1\!-\!11) &\mid (-\!11\!-\!1) \ (1\!-\!1\!-\!11)\\
	(-\!222\!-\!2) \mid (2\!-\!2\!-\!22) &\mid (0000) \mid (0000)\\
	(00) \mid (00) \mid (-\!44) \mid (4\!-\!4) &\mid (00) \mid (00) \mid (00) \mid (00)\\
	0 \mid 0 \mid 0 \mid 0 \mid 0 \mid 0 \mid -\!8 \mid 8 &\mid 0 \mid 0 \mid 0 \mid 0 \mid 0 \mid 0 \mid 0 \mid 0\\
	0 \mid 0 \mid 0 \mid 0 \mid 0 \mid 0 \mid 0 \mid -\!16 &\mid 0 \mid 0 \mid 0 \mid 0 \mid 0 \mid 0 \mid 0 \mid 0 
\end{align*}

Теперь делим каждый элемент на $16$:
$$
0000000(-\!1)00000000
$$

Сумма значения должна равняться $(-1)^{f[0]}$. В данном случае $-1$.
Проверка:
$$
0 \cdot 7 + (-1) + 0 \cdot 8 = -1.
$$

Умножим на 8 и получим коэффиценты статистической структуры:
$$
0000000(-\!8)00000000.
$$

\prim Круглые скобки необходимы лишь для наглядной иллюстрации блоков, с которыми работаем, в работе их указывать не нужно.