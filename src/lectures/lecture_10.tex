\section {Матроиды}
Рассмотрим $n$-мерные вектора над некоторым полем, например над $\R$. Множество таких векторов образует векторное пространство размерности $n$.\par
Некоторое семейство векторов $V^{\downarrow}_1,\ldots, V^{\downarrow}_k$ называется линейно независимым, если $\overline{\exists}$ набор коэффициентов $a_1, \ldots, a_k \in \R | \exists a_i \neq 0$ для : \par $a_1V^{\downarrow}_1 + \ldots + a_kV^{\downarrow}_k = 0^{\downarrow}$.\\ 

\prok 
\begin{enumerate}
\item Если $X$ - линейно независимое семейство векторов, то $Y \subset X$ также ЛНЗ (линейно независимо) 
\item Если $A$ - некоторое семейство векторов и $I_1;I_2$ - линейно независимые подмножества в $A$, максимальное по включению (т.е.$\forall V^{\downarrow} \in A \backslash I_i | I_i\cup \{ V^{\downarrow} \} $) становится линейно зависимым семейством, то $|I_1| = |I_2|$.
\item Если $I_1,I_2$ - ЛНЗ семейства и $|I_1| > |I_2|$, то $\exists V^{\downarrow} \in I_1 \backslash I_2 $ такой, что:\par 
$I_2 \cup \{ V^{\downarrow} \} $- ЛНЗ семейство (дополнение до базиса).                                                             
\end{enumerate}

\note Исходя из этих свойств получается много полезных результатов линейной алгебры. Возникает естественный вопрос: можно ли обобщить это понятие, выбирая указанные свойства в качестве аксиом. 

\opr $\sqsupset E $ - конечное непустное множество. $Y$ - некоторое семейство подмножества множества $E$ (то есть $Y \subseteq 2^E$). Причем, если $A \in Y$ и $ B \subseteq A$, то $ B \in Y$. Тогда пара $(E;Y)$ называется наследственной системой подмножеств.

\opr Элементы из $Y$ будем называть независимыми множествами. Остальные подмножества $A \subset E | A \notin Y $ назовем зависимыми.

\opr Наследвственную систему подмножеств называют матроидом, если $ \forall A \subset E$ и $ \forall I_1,I_2$ максимальным по включению независимых подмножеств множества $A$ верно: $|I_1| = |I_2|$.

\oprk Наследственная система подмножеств называется матроидом $\Leftrightarrow \forall I_1,I_2 \in Y : |I_1|>|I_2| \Rightarrow \exists e \in I_1 \backslash I_2 | I_2 \cup \{ e \} \in Y$.\\

\example 
\begin{enumerate}
\item $\sqsupset A$ - матрица размеров $n \times m$ ($n$ - строки, $m$ - столбцы).
$A = (A^{\downarrow}_1, \ldots , A^{\downarrow}_m)$ над $R$. Положим $ E = \{ A^{\downarrow}_1,\ldots,A^{\downarrow}_m \}$ - множество вектор стлобцов, $ \{ A_{i1}, \ldots, A_{ik} \} \in Y \Leftrightarrow$ семейство $ \{ A^{\downarrow}_{i_1}, \ldots, A^{\downarrow}_{i_k} \} $ ЛНЗ.
Матроид $(E;Y)$. Такие матроиды называют матричными. 

\item $E$ - конечное непустое множество и $0 \leqslant k \leqslant |E|. \\\sqsupset A \subseteq E \Leftrightarrow |A| < k $. $ (E;Y) $ - матроид, который называется  $k$ - однородным матроидом. При $k = |E|$, такой матроид называют дискретным. 
\end{enumerate}

\opr $\sqsupset M = (E;Y)$ - матроид. $ \sqsupset A \subset E$, так как по определению $\forall I_1,I_2$ - максимальный по включению независимых подмножеств в множестве $A$. $|I_1| = |I_2|$, то корректно определить ранг $A(\rho(A))$, который равен этой мощности (т.е. $\rho(A) = |I_1| = |I_2|$) $A \subset E $ независимо  $\Leftrightarrow |A| = \rho(A).$\par$ \rho(E) $ - ранг матриоида. Таким образом определена ранговая функция:\par
$\tilde{\rho}:2^E \rightarrow N_0 : \forall A \subseteq E \;\;\; \tilde\rho(A) = \rho(A)$

Вопрос: всякая ли функция $\tilde\rho$ является ранговой, то есть $\exists$ матроид $M$ такой, что $\tilde\rho:2^E \rightarrow N_0$ является ранговой функицей этого матроида.

\thr Функция $\rho : 2^E \rightarrow N_0$, где $E$ - конечное множество (непустое), является ранговой фунцией матроида $(E;Y) \Leftrightarrow$
\begin{enumerate}
\item $\forall A \subseteq E \;\;\;  \rho(A) \leqslant (A)$
\item Если $ A \subset B \subseteq E$, то $\rho(A) < \rho(B)$
\item $\forall A,B \subseteq E$ выполняется:
$\rho(A \cup B) + \rho(A \cap B) \leqslant \rho(A) + \rho(B)$
\end{enumerate} 

\note Для матричного матроида $M$ над матрицей $A$ имеем \\ $\rho(M) = rank(A)$. Таким образом матроид можно рассматривать как пару $(E;\rho)$, где $E$ конечное множество и $\rho:2^E \rightarrow N_0$ - функция удовлеторяющая условиям теоремы.

\opr $\sqsupset M_1 = (E_1;Y_1)$ и $ M_2 = (E_2;Y_2)$ - два матроида. \\$\phi : E_1 \rightarrow E_2$ - биекция. $\forall X \leqslant E$ положим $\phi(X) = \{\phi(x) : x\in X\}$. \\Отображение $\phi$ называется изоморфизмом матроидов \\ $M_1,M_2 \Leftrightarrow ((x\in Y_1) \Leftrightarrow (\phi(x) \in Y_2))$. Таким образом при изоморфизме независимые множества переходят в независимые. \\

\example $ \exists $ всего 4 неизоморфных матроида из двух элементов $E=\{1,2\}$
\begin{enumerate}
\item $Y = \{\varnothing\}$
\item $Y = \{ \varnothing ; \{ 1 \} ; \{ 2 \} \}$
\item $Y = \{ \varnothing ; \{ 1 \} ; \{ 2 \} ; \{ 1 ; 2 \} \}$
\item $Y = \{ \varnothing ; \{ 1 \} \} $
\end{enumerate}

\note Матроид изомофный матричным, называется матричным.
$ \sqsupset M = (E;Y)$ - наследственная система подмножест и для каждого $e\in E$ задан вес $\omega(e) \geqslant 0$. Требуется найти независимое подмножество с максимальным весом.\par
Вес подмножества $A\subseteq E |\; \omega(A) = \sum\limits_{a\in A}(\omega(a))$ \\

Схема решений для жадного алгоритма:\par
\quad начало:\par
\quad пока $E \neq 0$ делать \par
\qquad \{ пусть $e\in E $ элемент с наибольшим весом. Удалим $e$ из $E$;\par
\qquad если $I \cup \{ e \} $ - независимо, то $I := I \cup \{ e \} $\par
\qquad \}

\utv $\sqsupset M = (E;Y) $ наследственные системы подмножеств.\\ \par
\quad Эквивалентно:
\begin{enumerate}
\item $M$ - матроид.
\item При $\forall$ наборе весов на $E$ жадный алгоритм дает оптимальное решение.
\item Если $I_p;I_{p+1} \in Y$, где $|I_p| = p$ и $|I_{p+1}| = p + 1$, то \par $\exists e\in I_{p+1}\ I_p | I_p \cup \{ e \} \in Y $.\\
\end{enumerate}

\example(жадных алгоритмов)\par
 \begin{enumerate}
 \item Алгоритм Дейкстры
 \item Алгоритм Прима
 \item Алгоритм Хаффмана
 \end{enumerate}
