\section{Эквивалентность функций относительно групп преобразований.}

\subsection{Группы инерции}

\opr 
Подстановкой непустого множества M называют любое биективное отображение M на себя. При известном n будем обозначать f($x_1; \dotsc, x_n$)=f($\vec{x}$).

\opr 
Пусть f($\vec{x}$) и h($\vec{x}$) - функции из $F_k(n)$ и G - произвольная группа подстановок множества $\Omega_k^n$.Говорят ,что f эквивалентно h относительно группы G, если существует подстановка g $\in$ G | любого набора $\vec{\alpha} \in \Omega_k^n$ выполняется: $f(\vec{\alpha})=h(g(\vec{\alpha}))$. Обозначение : $f \stackrel{G}{\sim} h$.
 
\utv \\
1) $f \stackrel{G}{\sim} f$ \\
2) $f \stackrel{G}{\sim} h \Leftrightarrow h \stackrel{G}{\sim} f$ \\
3)$f \stackrel{G}{\sim} h;h \stackrel{G}{\sim} r \Rrightarrow f \stackrel{G}{\sim} r$\\

$\stackrel{G}{\sim}$ - отношение эквтвалентности.\\

Таким образом $F_k(n) $ разбивается на классы эквивалентности.Класс ,содержащий функию f, будем называть $[f]_G$. \\
Очевидно $1\leq|[f]_G\leq|G|$.\\


\opr 
Функция f($\vec{x}$) $\in F_k(n) $ - называется инвариантной относительно подстановки $g \in G < S_{\Omega_k^n}$ , если $f(g(\vec{x})) = f(\vec{x})$, относительно группы G,если она инвариантна относительно любой подстановки из этой группы.

\utv \\
Множество подстановок g $\in$ G, относительно которыйх функция f инвариантна образует подргуппу в группе G.

\opr 
Подргуппа ,определенная в утверждении, несет название : Группа инерции -функции f в группе G. Обозначение: $I_G(f)$.

\thr 
Если $f\in F_k(n) и G<S_(\Omega^k)$, то $|[f]_G|$=$\frac{|G|}{|I_G(f)}$

\opr 
Орбитой группы подстановок G<$S_(\Omega^k)$, содержащей элемент $\alpha$, называется множество $\bigtriangleup_\alpha$ = $\{ \beta \in S_(\Omega^k)|\exists g\in G;\beta =g(\alpha)\}$


\utv
f инвариантна относительно группы G $\Leftrightarrow$ на элементах каждоой орбиты она принимает постоянные значения. То есть $\forall p \in \bigtriangleup; f(\beta) = f(\alpha)$.

\conseq
G- транзитивна $\Rightarrow$ f инвариантна $\Rightarrow$ f $\equiv$ const

\conseq
Число функций k-значной логики инвариантно относительно группы G ровно $k^{\nu(G)}$- число орбит группы G.\\

1)Группа подстановок координат векторов $\alpha \in \Omega_k^n $
$g_s(a_1, \dots , a_n) $ = $ a_i1;\dots ;a_in$ | в соответсвии с перестановкой 

$\begin{pmatrix}
  1;& \dots & n\\
  \dots & \dots & \dots\\
  i_1; & \dots & i_n
  
  
\end{pmatrix}$\\

Обозначение: $ S_n $.\\

2)Группа сдвигов: $ \sum_n $. Пусть $ \alpha$ = $(a_1;\dots;a_n) \in \Omega_k^n $. Тогда $ \sum_n = \{g_\alpha|g_\alpha(C_1;\dots;C_n)$=$(C_1+a_1;\dots;C_n+a_n);\alpha и \vec{c} \in \Omega_k^n\}\\ $ \\
Суммирование ведется по mod k.\\
3)Группа Джевонса $Q_n$\\
$Q_n = < \sum_n;S_n>$\\
4)GL(n;k)- полная линейная группа; Пусть A - невырожденная матрица размеров n $\times$n над $ Z_k$.Тогда GL(n;k)=$\{g_A|g_A(a_1;\dots;a_n)+(a_1;\dots;a_n)\cdotp A; A \in (Z_k|_(n \times n) ^A!!!!!!\}$
5)Полная афинная группа AGL(n;k)=<GL(n;k);$\sum_n$>//
Диаграмма вложения группы!!!!!!\\



$\sum_n;Q_n;AGL(n;k) $ - транзитивны $\rightarrow$ инвариантны относительно них только const.\\
Орбитами группы $S_n$ являются множества векторов одинакового веса. Число функций одинакового веса инвариантны относительно$ S_n$ ровно $k^\frac {k+n-1}{k-1}$.

\thr
Орбитами группы GL(n;k) являются все множества $M_\alpha=\{(a_1;\dots;a_n)\in \Omega_k^n|НОД(a_1;\dots;a_n;k)=d\},где d|k$.
                   
\cled
Число функций n-значной логики (от n-переменных ),инвариантных относительно GL(n;k) равно $k^{\nu(k)},где \nu(k)$-число делителей k.\\


\thr
Пусть $T_k(n)$-множество всех функций k-значной логики с тривиальной группой инерции в AGL(n;k). Тогда $  \lim_{n\to\infty} \frac{|T_n(k)|}{|F_k(n)|}  =1$.\\

\proof
1)Оценим сверху число неподвижных точек нетождественного фаинного преобразования $g \in AGL(n;k)$.  Рассмотрим уравнение $g(\vec{x})=\vec{x}A+\alpha;\alpha \in Z_k^n;$ A-невырожденная матрица над $Z_k;$ В матричном виде уравнение перепишется следующим образом:\\
$\vec{x}(E-A)=\alpha $\\
$\cdot$ Если A=E ,то $\nexists$ решений , так как $\vec{\alpha}\neq \vec{0}$ (g - нетождественное преобразование)







