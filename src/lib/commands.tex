% ======================
% Данный файл должен содержать команды для упрощения написания текста
% Ниже вы найдете некоторый перечень комманд, которые я создал. На основе их можно создавать свои.
% ======================

% Красивые буквы
\newcommand{\N}{\mathbb{N}}
\newcommand{\Q}{\mathbb{Q}}
\newcommand{\Z}{\mathbb{Z}}
\newcommand{\K}{\mathbb{K}}
\newcommand{\R}{\mathbb{R}}
\renewcommand{\C}{\mathbb{C}}
\newcommand{\X}{\mathcal{X}}
\newcommand{\hatv}[1]{\overset{\wedge}{\mathstrut#1}}

% Определения, утверждения, теоремы, замечания
\newcommand{\opr}{\emph{\textbf{Определение.}} }
\newcommand{\opri}{\emph{\textbf{\underline{Определение.}}} }
\newcommand{\utv}{\emph{\textbf{Утверждение.}} }
\newcommand{\utvi}{\emph{\textbf{\underline{Утверждение.}}} }
\newcommand{\thr}{\emph{\textbf{Теорема.}} }
\newcommand{\thri}{\emph{\textbf{\underline{Теорема.}}} }
\newcommand{\proof}{\emph{Док-во:} }
\newcommand{\lem}{\emph{\textbf{Лемма.}} }
\newcommand{\lemi}{\emph{\textbf{\underline{Лемма.}}} }
\newcommand{\note}{\emph{\textbf{Замечание.}} }
\newcommand{\notei}{\emph{\textbf{\underline{Замечание.}}} }
\newcommand{\conseq}{\emph{\textbf{Следствие.}} }
\newcommand{\example}{\emph{Пример.} }
\newcommand{\examplei}{\emph{\textbf{Пример.}} }
\newcommand{\prop}{\emph{Свойства:} }
\newcommand{\prim}{\emph{Примечание:} }

% Команда для отображения
\newcommand{\map}[3]{\ensuremath{#1: #2 \rightarrow #3}}

% Команда для красивого модуля
\newcommand{\Mod}[1]{\ (\mathrm{mod}\ #1)}